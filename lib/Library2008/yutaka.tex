\begin{verbatim}
#include<iostream>
#include<cfloat>
#include<cassert>
#include<cmath>

using namespace std;

#define EPS (1e-10)
#define equals(a, b) (fabs((a) - (b)) < EPS )

class Point{
    public:
    double x, y;
    
    Point ( double x = 0, double y = 0): x(x), y(y){}
    
    Point operator + ( Point p ){ return Point(x + p.x, y + p.y); }
    Point operator - ( Point p ){ return Point(x - p.x, y - p.y); }
    Point operator * ( double a ){ return Point(a*x, a*y); }

    double abs() { return sqrt(norm());}
    double norm() { return x*x + y*y; }

    bool operator < ( const Point &p ) const {
        return x != p.x ? x < p.x : y < p.y;
    }

    bool operator == ( const Point &p ) const {
        return fabs(x-p.x) < EPS && fabs(y-p.y) < EPS;
    }
};

typedef Point Vector;

class Segment{
    public:
    Point p1, p2;
    Segment(Point s = Point(), Point t = Point()): p1(s), p2(t){}
};

double norm( Vector a ){
    return a.x*a.x + a.y*a.y;
}

double abs( Vector a ){
    return sqrt(norm(a));
}

double getDistance( Vector a, Vector b ){
    return abs(a - b); 
}

double dot( Vector a, Vector b ){
    return a.x*b.x + a.y*b.y;
}

double cross( Vector a, Vector b ){
    return a.x*b.y - a.y*b.x;
}

Point project( Segment s, Point p ){
    Vector base = s.p2 - s.p1;
    double t = dot(p - s.p1, base)/norm(base);
    return s.p1 + base*t;
}

Point reflect( Segment s, Point p ){
    return p + (project(s, p) - p)*2.0;
}

// verified by uoa2062
bool isOnSegment( Point a, Point b, Point c){
    if ( a == c || b == c ) return true;
    return (abs(a-c) + abs(c-b) < abs(a-b) + EPS );
}

bool isOrthogonal( Vector a, Vector b ){
     return equals( dot(a, b), 0.0 );
}

bool isOrthogonal( Point a1, Point a2, Point b1, Point b2 ){
    return isOrthogonal( a1 - a2, b1 - b2 );
}

bool isOrthogonal( Segment s1, Segment s2 ){
    return equals( dot(s1.p2 - s1.p1, s2.p2 - s2.p1), 0.0 );
}

bool isParallel( Vector a, Vector b ){
    return equals( cross(a, b), 0.0 );
}

bool isParallel( Point a1, Point a2, Point b1, Point b2){
    return isParallel( a1 - a2, b1 - b2 );
}

bool isParallel( Segment s1, Segment s2 ){
    return equals( cross(s1.p2 - s1.p1, s2.p2 - s2.p1), 0.0 );
}

static const int COUNTER_CLOCKWISE = 1;
static const int CLOCKWISE = -1;
static const int ONLINE_BACK = 2;
static const int ONLINE_FRONT = -2;
static const int ON_SEGMENT = 0;

// EPS can be 0
// need to check for 920, 833, 866
int ccw( Point p0, Point p1, Point p2 ){
    Vector a = p1 - p0;
    Vector b = p2 - p0;
    if ( cross(a, b) > EPS ) return COUNTER_CLOCKWISE;
    if ( cross(a, b) < -EPS ) return CLOCKWISE;
    if ( dot(a, b) < -EPS ) return ONLINE_BACK;
    if ( norm(a) < norm(b) ) return ONLINE_FRONT;
    return ON_SEGMENT;
}

// intersect Segment p1-p2 and Segment p3-p4 ?
bool isIntersect(Point p1, Point p2, Point p3, Point p4){
    return ( ccw(p1, p2, p3) * ccw(p1, p2, p4) <= 0 &&
             ccw(p3, p4, p1) * ccw(p3, p4, p2) <= 0 );
}

// intersect Segment s1 and Segment s2 ?
// verified by 920, 833, 866, uoa2062
bool isIntersect(Segment s1, Segment s2){
    return isIntersect(s1.p1, s1.p2, s2.p1, s2.p2);
}

// verified by 920, 833, uoa2062
Point getCrossPoint(Segment s1, Segment s2){
    assert( isIntersect(s1, s2) );
    Vector base = s2.p2 - s2.p1;
    double d1 = abs(cross(base, s1.p1 - s2.p1));
    double d2 = abs(cross(base, s1.p2 - s2.p1));
    double t = d1/(d1 + d2);
    return s1.p1 + (s1.p2 - s1.p1)*t;
}


int main(){

    return 0;
}




#include<stdio.h>
#include<iostream>
#include<vector>
#include<queue>
using namespace std;
#define REP(i, e) for ( int i = 0; i < (int)e; i++ )
#define MAX 100
#define MAXK 10
#define INFTY (1<<21)

class Edge{
    public:
    int v, cost;
    Edge(){}
    Edge( int v, int cost): v(v), cost(cost){}
};

class State{
    public:
    int u, cost;
    State(){}
    State(int u, int cost):u(u), cost(cost){}

    bool operator < ( const State &s) const{
        if ( cost == s.cost ) return u > s.u;
        return cost > s.cost;
    }
};

vector<Edge> G[MAX];
int D[MAX][MAXK];

int size, source, target, K;

void kthDijkstra(){
    priority_queue<State> PQ;
    int L[MAX]; // level

    REP(i, size) REP(k, K){
        D[i][k] = INFTY;
    }
    REP(i, size) L[i] = 0;

    D[source][0] = 0;
    PQ.push(State(source, 0));

    State s;
    while(!PQ.empty()){
        s = PQ.top(); PQ.pop();
        if ( L[s.u] >= K ) continue;
        D[s.u][L[s.u]++] = s.cost;
        REP(i, G[s.u].size() ) if ( L[G[s.u][i].v] <= K ){
            PQ.push(State(G[s.u][i].v, s.cost + G[s.u][i].cost ));
        }
    }
}

int compute(){
    kthDijkstra();
    if ( D[target][K-1] == INFTY ) return -1;
    else return D[target][K-1];
}

bool input(){
    int m, s, t, c;
    scanf("%d %d", &size, &m);
    if ( size == 0 && m == 0 ) return false;
    scanf("%d %d %d",  &source, &target, &K);
    source--; target--;
    REP(i, size) G[i].clear();
    REP(i, m){
        scanf("%d %d %d", &s, &t, &c);
        s--; t--;
        G[s].push_back(Edge(t, c));
    }

    return true;
}

int main(){
    while(input()) printf("%d\n", compute());
    return 0;
}
\end{verbatim}